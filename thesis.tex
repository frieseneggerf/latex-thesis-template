\documentclass[english,12pt,twoside]{article}
\usepackage{palatino}
\usepackage[T1]{fontenc}
\usepackage{textgreek, textcomp, multirow, acronym, array, svg, longtable, tabularx}
\usepackage[l3]{csvsimple}

\usepackage{geometry}
\geometry{a4paper, top=2.5cm, bottom=2.5cm, left=2.5cm, right=2.5cm, bindingoffset=0cm}
			
\usepackage{graphicx}
\graphicspath{{graphics}}

\usepackage{fancyhdr}
\setlength{\headheight}{20pt}
\setlength{\footskip}{30pt}
\pagestyle{fancy}
\fancyhf{}
\fancyfoot[LE,RO]{\thepage}
\fancyhead[LE,RO]{\leftmark}
			
\usepackage[backend=biber,
			style=apa,
			sorting=nyt,
			url=false
			]{biblatex}
\addbibresource{references.bib}

\usepackage[font=small,labelfont=bf]{caption}

\usepackage{float}
\restylefloat{table}
\floatstyle{plaintop}
\restylefloat{table}

\usepackage{parskip}
\usepackage{setspace}
\onehalfspacing

%==== Function \formatDNA{} takes string and separates it into groups of 3 characters======
%Useful for making dna sequences readable & look good (ofc no relation to biological readingframe)
\makeatletter%
\newcounter{@count}%
\newcommand{\formatDNA}[1]{%
	\setcounter{@count}{0}%
	\@tfor\base:=#1\do{%
		\base% 
		\stepcounter{@count}%
		\ifnum\value{@count}=3
		\space%
		\setcounter{@count}{0}%
		\fi%
	}%
}%
\makeatother%
%==================================================================


%Uncomment line below to mark pdf as draft:
%\fancyhf[HC, FC]{-DRAFT-}

%Enter your data for pdf metadata
\usepackage[pdftex, hidelinks,
pdfauthor={A. Uthor},
pdftitle={Example Title: Investigations on LaTeX Templates},
pdfsubject={Bachelor's Thesis},
pdfkeywords={topic 1, topic 2, topic x, bachelor thesis},
]{hyperref}


\begin{document}
	\pagenumbering{roman}
	
	\begin{titlepage}
	\begin{center}
		\uppercase{
			Ludwig-Maximilians-Universität München\\
			\vspace{1mm}
			Department of Pharmacy}
		\vspace{18mm}
		
		\includesvg[height=45mm]{seal_lmu.svg}
		\vspace{18mm}
		
		\begin{LARGE}
			\textbf{Writing a Thesis - Investigations on LaTeX Templates}
		\end{LARGE}
		\vspace{10mm}
		
		\textbf{A. Uthor}\\
		Matr.-Nr.: 12345
		\vspace{10mm}
		
		Bachelor's Thesis\\
		in Pharmaceutical Sciences
		\vfill
		
		\begin{tabular}{ r l }
			Supervisor:  & B. Igboss \\
			Advisor:  & A. D. Visor \\
			Submission Date: & 01. April 2024 \\
		\end{tabular}
	\end{center}
\end{titlepage}
	\newpage
	
	I declare that I wrote this thesis on my own using only the listed sources and references.
\vspace{2cm}

City, 01. April 2024 \hspace{2cm} Your Name
	\newpage
	
	\section*{Abstract\markboth{ABSTRACT}{}}
\addcontentsline{toc}{section}{Abstract}
Describing all the \textbf{interesting} stuff done and found...

	\newpage
	
	\renewcommand{\contentsname}{Table of Contents}
	\addcontentsline{toc}{section}{\contentsname}
	\tableofcontents
	\newpage
	
	\pagenumbering{arabic}
	
	\section{Introduction}
Writing is hard sometimes (\cite{HinzEtAl2024}).
	\newpage
	
	\section{Materials and Methods}
\subsection{Materials}
\subsubsection{Chemicals and Kits}
All chemicals used were of analytical grade.

\begin{table}[H]
	\renewcommand{\arraystretch}{1.2}
	\centering
	\caption{List of chemicals and kits}
	\label{table:chemicals-kits}
	\begin{tabular*}{\linewidth}{@{\extracolsep{\fill}}l c m{5cm} } 
		\hline
		\textbf{Name} & \textbf{Product \#}  & \textbf{Supplier} \\ 
		\hline
		Chemical A & K1234 & Delta \newline Burlington, US-MA \\
		Kit B\textregistered\ 100 & K4321 & Companyy \newline Karlsruhe, DE-BW \\
		\hline
	\end{tabular*}
\end{table}

\subsubsection{qPCR Primers}
All primers are DNA oligos at 100\,µM.

{
\renewcommand{\arraystretch}{1.2}
\centering

\begin{longtable}{ 
		>{\centering\let\newline\\\arraybackslash\hspace{0pt}}m{2cm} 
		>{\centering\let\newline\\\arraybackslash\hspace{0pt}}m{3cm} 
		m{9.7cm} }
	\caption{List of primer sequences used for qPCR.}
	\label{table:qpcr-primer}\\
	\hline
	\textbf{Gene} & \textbf{Direction} & \textbf{Sequence 5'\textrightarrow3'} 
	\csvreader[head to column names]{data/example_primer.csv}{}{
		\\\hline \textit{\gene}  & fwd \newline rev & {\ttfamily \expandafter\formatDNA\expandafter{\seqfwd} \newline \expandafter\formatDNA\expandafter{\seqrev}}
	}
	\\\hline
\end{longtable}
}


\newpage
\subsection{One Method}
Some \ac{PCR}...

\begin{table}[H]
	\renewcommand{\arraystretch}{1.2}
	\centering
	\begin{tabular}{ c c c }
		\hline
		\textbf{Time [s]} & \textbf{Temperature [°C]} & \textbf{Step} \\
		\hline
		120  & 50 & \multirow{3}{*}{hold}\\
		20 & 50\textrightarrow 95 & \\
		120 & 95 & \\
		\hline
		3  & 95 & \multirow{5}{*}{PCR (40\texttimes)}\\
		17 & 95\textrightarrow 58 & \\
		5 & 58 & \\
		1 & 58\textrightarrow 60 & \\
		25 & 60 & \\
		\hline
		16 & 60\textrightarrow 95 & \multirow{6}{*}{melt curve}\\
		15 & 95 & \\
		16& 95\textrightarrow 60 & \\
		60 & 60 & \\
		117 & 60\textrightarrow 95 & \\
		16 & 95 & \\
		
		\hline
	\end{tabular}
	\caption{qPCR protocol}
	\label{table:qpcr-protocol}
\end{table}

\subsection{Another Method}

	\newpage
	
	\input{src/03-results.tex}
	\newpage
	
	\input{src/04-discussion.tex}
	\newpage
	
	\input{src/05-references.tex}
	\newpage
	
	\section*{Abbreviations\markboth{ABBREVIATIONS}{}}
\addcontentsline{toc}{section}{Abbreviations}

\begin{acronym}
	\acro{PCR}{polymerase chain reaction}
\end{acronym}
	\newpage
	
	\section*{Acknowledgments\markboth{ACKNOWLEDGMENTS}{}}
\addcontentsline{toc}{section}{Acknowledgments}
%When adding an unnumbered section using "\section*{}" you can add it to the toc manually using the above command
	%\newpage
	
	%\input{another_chapter.tex}
	% ...
\end{document}